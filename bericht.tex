\documentclass[12pt, a4paper, twoside]{article}
\usepackage[utf8]{inputenc}
\usepackage{fancyhdr}
\usepackage{graphicx}
\usepackage{caption}
\usepackage{subcaption}
\usepackage[margin=1in,footskip=0.25in]{geometry}
\usepackage{pgf}
\usepackage[absolute, overlay]{textpos}
\usepackage{pdfpages}
\usepackage[siunitx]{circuitikz}
\usepackage{adjustbox}

\setlength{\TPHorizModule}{1.0 pt}
\textblockorigin{\paperwidth}{0.0 pt}

\pagestyle{fancy}
\fancyhf{}
\lhead{Stromkreise \& Glühlampchen}
\rhead{Physik Bericht}
\begin{document}
    \pgfmathwidth{"\\\\ J. Dubois \& A. Huillet\\ MNG Rämibühl\\"}
    \begin{textblock}{\pgfmathresult}[1, 0](0, 0)
    \noindent
    \\\\ J. Dubois \& A. Huillet\\ MNG Rämibühl\\ 8001 Zürich
    \end{textblock}
    \begin{titlepage}

    \begin{center}
        \vspace*{8cm}
        \Huge
        \textbf{Stromkreise \& Glühlämpchen}
 
        \vspace{0.5cm}
        \LARGE
        Physik Kurzbericht  
    \end{center}
    \vspace*{8cm}
    \normalsize
    \hspace*{2cm}Physik Bericht mit Bezug auf das Praktikum vom 24. März 2022\\\\
    \hspace*{2cm}Zürich, 7. April 2022
    
             
 \end{titlepage}

    \newpage
    \includepdf[pages=-]{aufgabenstellung.pdf}
    \section{Einleitung}
    \newpage
    \section{Theorie}
    \newpage
    \section{Experiment}
    \newpage
    \section{Messungen}     
    \subsection{Messung A}
    Ein einfacher Stromkreis wurde gebaut. Dieser wird für die Messungen A, B und C\\ benutzt:\\
    \newline
    \begin{center}
        \begin{adjustbox}{scale=0.4}
            \begin{circuitikz} \draw
                (0,10) to[vsource, i=\LARGE{$I$}, l_=\LARGE{$U$}] (0,0)
                (0,10) to[lamp, i=\LARGE{$I$}] (16,10)--(16,0)--(0,0)
                ;
            \end{circuitikz}        
        \end{adjustbox}
        \end{center}
    \subsection{Messung B}
    Bei Messung B wurden für zehn verschiedenen Spannungswerte den Strom durch das Lämpchen gemessen:\\
    \begin{center}
        \begin{tabular}{l|r}
            \textbf{Spannung U[V]} & \textbf{Stromstärke I[mA]}\\
            \hline
            3.65 & 39.50\\
            5.11 & 48.30\\
            8.33 & 64.60\\
            10.37 & 73.60\\
            12.50 & 82.20\\
            15.54 & 93.60\\
            17.62 & 100.70\\
            19.74 & 107.70\\
            23.80 & 120.00\\
            24.85 & 123.00
        \end{tabular}
    \end{center}
    \newpage
    \subsection{Messung C}
    In dieser Messung wurde das gleiche Verfahren wie bei Messung B bei zwei weiter Lämpchen angewendet.\\\\
    \textbf{Messung für Lämpchen 2:}\\
    \begin{center}
        \begin{tabular}{l|r}
            \textbf{Spannung U[V]} & \textbf{Stromstärke I[mA]}\\
            \hline
            3.65 & 39.10\\
            5.11 & 48.40\\
            8.33 & 64.00\\
            10.37 & 73.30\\
            12.50 & 81.80\\
            15.54 & 93.60\\
            17.62 & 101.00\\
            19.74 & 107.90\\
            23.80 & 120.70\\
            24.85 & 123.90
        \end{tabular}
    \end{center}
    \vspace{1cm}
    \textbf{Messung für Lämpchen 3:}\\
    \begin{center}
        \begin{tabular}{l|r}
            \textbf{Spannung U[V]} & \textbf{Stromstärke I[mA]}\\
            \hline
            3.65 & 39.30\\
            5.11 & 48.50\\
            8.33 & 64.10\\
            10.37 & 73.30\\
            12.50 & 81.70\\
            15.54 & 93.40\\
            17.62 & 100.10\\
            19.74 & 107.10\\
            23.80 & 119.60\\
            24.85 & 122.80
        \end{tabular}
    \end{center}
    \subsection{Messung D}
    Folgender Stromkreis wurde aufgebaut:
    \begin{center}
        \begin{adjustbox}{scale=0.4}
            \begin{circuitikz} \draw
                (0,10) to[vsource, i=\LARGE{$I_{tot}$}, l_=\LARGE{$U_{tot}$},invert] (0,0)
                (0,10) -- (4, 10)
                (4,10) to[lamp, l=\LARGE{$U_1$}, i=\LARGE{$I_1$}] (8,10)
                (8,10) to[lamp, l=\LARGE{$U_2$}, i=\LARGE{$I_2$}] (12,10)
                (12,10)--(16,10) -- (16,0) -- (0,0)
                ;
            \end{circuitikz}  
        \end{adjustbox}
    \end{center}
    Folgende Werte ergaben sich:\\
    \begin{center}
        \begin{tabular}{l|r}
            \textbf{U (Gesamtspannung)} & \textbf{I (Gesamtstrom)}\\
            \hline
            20.81 \textit{V} & 73.6 \textit{mA}\\
            \hline
        \end{tabular}
    \end{center}
    \vspace{1cm}
    \begin{center}
        \begin{tabular}{l|r}
            \textbf{U1 (Teilspannung 1)} & \textbf{I1 (Teilstrom 1)}\\
            \hline
            10.37 \textit{V} & 73.57 \textit{mA}\\
            \hline
        \end{tabular}
    \end{center}
    \vspace{1cm}
    \begin{center}
        \begin{tabular}{l|r}
            \textbf{U2 (Teilspannung 2)} & \textbf{I2 (Teilstrom 2)}\\
            \hline
            10.45 \textit{V} & 73.6 \textit{mA}\\
            \hline
        \end{tabular}
    \end{center}
    \subsection{Messung E}
    Folgender Stromkreis wurde aufgebaut:
    \begin{center}
        \begin{adjustbox}{scale=0.4}
        \begin{circuitikz} \draw
            (0,10) to[vsource,l_=\LARGE{$U_{tot}$}, i=\LARGE{$I_{tot}$}, invert] (0,0)
            (0,10)--(16/3,10)
            (16/3,10)--(16/3,12)
            (16/3,10)--(16/3,8)
            (16/3,12) to[lamp, i=\LARGE{$I_1$}, l=\LARGE{$U_1$}] (32/3,12)
            (16/3,8) to[lamp, i=\LARGE{$I_2$}, l=\LARGE{$U_2$}] (32/3,8)
            (32/3,12)--(32/3,10)
            (32/3,8)--(32/3,12)
            (32/3,10)--(16,10)
            (16,10)--(16,0)--(0,0)
            ;
        \end{circuitikz}
    \end{adjustbox}
    \end{center}
    Die Gesamtspanung $U_{tot}$ und der Gesamtstrom $I_tot$ sind bekannt:
        \[U_{tot} = 20.59 V\]
        \[I_{tot} = 220.6 mA\]
    \newline
    Man weiss schon, dass:
    \[U_1 = U_2 = U_{tot}\]
    \newline
    Das Multimeter zur Strommessung mass folgende Werte für $I_1$ und $I_2$:
        \[I_1 = 111.3 mA \]
        \[I_2 = 110.5 mA\]
    \subsection{Messung F}
    1)
    \begin{center}
        \begin{adjustbox}{scale=0.4}
            \begin{circuitikz}\draw
                (0,10) to[vsource, l_=\LARGE{$U_{tot}$}, i_=\LARGE{$I_{tot}$}] (0,0)
                (0,10) to[lamp, l=\LARGE{$U_1$}, i=\LARGE{$I_1$}] (16/3,10) 
                (16/3,10) to[lamp, l=\LARGE{$U_2$}, i=\LARGE{$I_2$}] (32/3,10)
                (32/3,10) to[lamp, l=\LARGE{$U_3$}, i=\LARGE{$I_3$}] (16,10)
                (16,10)--(16,0)--(0,0)
                ;
            \end{circuitikz}
        \end{adjustbox}
    \end{center}
    2)
    \begin{center}
        \begin{adjustbox}{scale=0.4}
            \begin{circuitikz}\draw
                (0,10) to[vsource, invert, l_=\LARGE{$U_{tot}$}, i_=\LARGE{$I_{tot}$}] (0,0)
                (0,10)--(4,10)
                (4,10)--(4,11)
                (4,10)--(4,9)
                (4,11) to[lamp, l=\LARGE{$U_1$}, i=\LARGE{$I_1$}] (8,11)
                (4,9) to[lamp, l_=\LARGE{$U_2$}, i_=\LARGE{$I_2$}] (8,9)
                (8,11)--(8,10)
                (8,9)--(8,10)
                (8,10) to[lamp, l=\LARGE{$U_3$}, i=\LARGE{$I_3$}] (16,10)
                (16,10)--(16,0)--(0,0)
                ;
            \end{circuitikz}
        \end{adjustbox}
    \end{center}
    3)
    \begin{center}
        \begin{adjustbox}{scale=0.4}
            \begin{circuitikz}\draw
                (0,10) to[vsource, invert, l_=\LARGE{$U_{tot}$}, i_=\LARGE{$I_{tot}$}] (0,0)
                (0,10) to[lamp, l_=\LARGE{$U_1$}, i_=\LARGE{$I_1$}] (8,10)
                (8,10) to[lamp, l_=\LARGE{$U_3$}, i_=\LARGE{$I_3$}] (16,10)
                (4,10.4)--(4,11)
                (4,11) to[lamp, l=\LARGE{$U_2$}, i=\LARGE{$I_2$}] (12,11)
                (12,11)--(12,10.4)
                (16,10)--(16,0)--(0,0)
                ;
            \end{circuitikz}            
        \end{adjustbox}
    \end{center}
    4)
    \begin{center}
        \begin{adjustbox}{scale=0.4}
            \begin{circuitikz}\draw
                (0,10) to[vsource, invert, l_=\LARGE{$U_{tot}$}, i_=\LARGE{$I_{tot}$}] (0,0)
                (0,10)--(4,10)
                (4,10) to[lamp, l=\LARGE{$U_2$}, i=\LARGE{$I_2$}] (12,10)
                (4,10)--(4,12)
                (4,10)--(4,8)
                (12,10)--(12,12)
                (12,10)--(12,8)
                (4,12) to[lamp, l=\LARGE{$U_3$}, i=\LARGE{$I_3$}] (12,12)
                (4,8) to[lamp, l=\LARGE{$U_1$}, i=\LARGE{$I_1$}] (12,8)
                (12,10)--(16,10)
                (16,10)--(16,0)--(0,0)
                ;
            \end{circuitikz}
        \end{adjustbox}
    \end{center}
    Unsere Vermutungen (Helligkeit):
    \begin{enumerate}
        \item 1, 1, 1
        \item 0, 0, 1
        \item $\frac{1}{2}$, 1, $\frac{1}{2}$
        \item 1, 1, 1
    \end{enumerate}
    \subsection{Messung G}
    Folgende Helligkeiten ergaben sich nach dem Bau der Schaltungen von \textbf{Messung F}:
    \begin{enumerate}
        \item $\frac{1}{3}$, $\frac{1}{3}$, $\frac{1}{3}$
        \item 0, 0, 1
        \item $\frac{1}{2}$, 1, $\frac{1}{2}$
        \item 1, 1, 1
    \end{enumerate}
    Folgende Messwerte wurden gemessen:
    \begin{center}
        \begin{tabular}{l|l|l|c|r|r}
            \textbf{Schaltschema} & \textbf{$U_1$} & \textbf{$U_2$} & \textbf{$U_3$} & \textbf{$I_{tot}$} & \textbf{$U_{tot}$}\\
            \hline
            \textbf{1} & 4.196 \textit{V} & 4.100 \textit{V} & 4.250 \textit{V} & 42.500 \textit{mA} & 12.530 \textit{V} \\
            \textbf{2} & 0 \textit{V} & 0 \textit{V} & 12.480 \textit{V} & 81.200 \textit{mA} & 12.480 \textit{V} \\
            \textbf{3} & 6,235 \textit{V} & 12.400 \textit{V} & 6.169 \textit{V} & 155.2 \textit{mA} & 12.400 \textit{V} \\
            \textbf{4} & 12.430 \textit{V} & 12.430 \textit{V} & 12.430 \textit{V} & --Kein Wert-- & 12.430 \textit{V}

        \end{tabular}
    \end{center}
    \newpage
    \section{Aufgaben}
    \newpage
    \section{Schlussfolgerungen}
    \newpage


\end{document}
